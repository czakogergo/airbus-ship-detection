\documentclass{article}

% if you need to pass options to natbib, use, e.g.:
% \PassOptionsToPackage{numbers, compress}{natbib}
% before loading nips_2016
%
% to avoid loading the natbib package, add option nonatbib:
% \usepackage[nonatbib]{nips_2016}

% \usepackage{nips_2016}

% to compile a camera-ready version, add the [final] option, e.g.:
\usepackage[final]{nips_2016}
\makeatletter
\let\@conferenceinfo\@empty
\makeatother



\usepackage[utf8]{inputenc} % allow utf-8 input
\usepackage[T1]{fontenc}    % use 8-bit T1 fonts
\usepackage{hyperref}       % hyperlinks
\usepackage{url}            % simple URL typesetting
\usepackage{booktabs}       % professional-quality tables
\usepackage{amsfonts}       % blackboard math symbols
\usepackage{nicefrac}       % compact symbols for 1/2, etc.
\usepackage{microtype}      % microtypography

\title{
Ship Detection in Satellite Images Using Deep Learning\\
\large Hajók detektálása műholdképeken mélytanulás segítségével
}



% The \author macro works with any number of authors. There are two
% commands used to separate the names and addresses of multiple
% authors: \And and \AND.
%
% Using \And between authors leaves it to LaTeX to determine where to
% break the lines. Using \AND forces a line break at that point. So,
% if LaTeX puts 3 of 4 authors names on the first line, and the last
% on the second line, try using \AND instead of \And before the third
% author name.

\author{
Sági Benedek \\
Budapest Műszaki és Gazdaságtudományi Egyetem \\
\texttt{benedek.sagi@edu.bme.hu}
\And
Czakó Gergő \\
Budapest Műszaki és Gazdaságtudományi Egyetem \\
\texttt{gergo.czako@edu.bme.hu}
}

\begin{document}
% \nipsfinalcopy is no longer used

\maketitle

\begin{abstract}
\textbf{Abstract}
The automatic analysis of satellite imagery plays a crucial role in maritime traffic monitoring, environmental protection, and security applications. This paper addresses the task defined in the Airbus Ship Detection Challenge, which aims to detect ships in high-resolution satellite images. The proposed approach employs a convolutional neural network-based segmentation model to identify ship locations using binary pixel-level masks. The model is trained in a supervised manner on annotated satellite images, where ship masks are provided in run-length encoded format. During training, separate training and validation sets are used to monitor generalization performance and prevent overfitting. The model is evaluated on training, validation, and test datasets using segmentation-specific metrics and loss functions. Experimental results demonstrate that the deep learning-based segmentation approach is effective for ship detection in complex and noisy visual environments, making it a suitable solution for maritime remote sensing applications.

\vspace{0.7em}
\textbf{Magyar kivonat}
A műholdfelvételek automatikus elemzése kulcsfontosságú szerepet tölt
be a tengeri forgalom megfigyelésében, a környezetvédelemben és a
biztonsági alkalmazásokban. A dolgozat az Airbus Ship Detection
Challenge keretében meghatározott feladatot vizsgálja, amelynek célja
hajók detektálása nagy felbontású műholdképeken. A bemutatott módszer
egy konvolúciós neurális hálózaton alapuló szegmentációs modellt
alkalmaz, amely bináris, pixel-szintű maszkok segítségével határozza
meg a hajók elhelyezkedését a képeken. A modellt felügyelt tanulási módszerrel, annotált műholdképeken
tanítottuk, ahol a hajókhoz tartozó maszkok futamhossz-kódolt
formátumban álltak rendelkezésre. A tanítás során külön tanító és
validációs adathalmazt használtunk az általánosítási képesség nyomon
követésére és a túltanulás elkerülésére. A modell teljesítményét
tanítási, validációs és teszt adathalmazokon értékeltük, szegmentációs
feladatokra jellemző mérőszámok és veszteségfüggvények alkalmazásával. A kísérleti eredmények azt mutatják, hogy a mélytanulás-alapú
szegmentációs megközelítés hatékonyan alkalmazható a hajók
detektálására összetett és zajos vizuális környezetben, így alkalmas
megoldást jelent a tengeri távérzékelési alkalmazások számára.
\end{abstract}



\section{Bevezető}
A műholdas távérzékelés napjainkban kulcsszerepet játszik a Föld
felszínének folyamatos megfigyelésében. A nagy felbontású műholdképek
lehetővé teszik a tengeri forgalom elemzését, a hajózási útvonalak
monitorozását, valamint az illegális tevékenységek – például az
engedély nélküli halászat – felderítését. Ezek az alkalmazások
megkövetelik a hajók pontos és megbízható automatikus detektálását,
amely jelentős kihívást jelent a változatos környezeti feltételek
miatt.

A hajók megjelenése a műholdképeken rendkívül heterogén: méretük,
alakjuk, orientációjuk és kontrasztjuk jelentősen eltérhet, továbbá a
háttér gyakran zajos, hullámzó vízfelszínt vagy part menti struktúrákat
tartalmaz. A hagyományos képfeldolgozási módszerek jellemzően nem
bizonyulnak elég robusztusnak ilyen komplex vizuális környezetben,
ezért az utóbbi években a mélytanulás-alapú megközelítések kerültek
előtérbe.

Jelen dolgozat az Airbus Ship Detection Challenge keretében megadott
feladatot dolgozza fel, amely hajók pixel-szintű szegmentációját tűzi
ki célul műholdképeken. A munka során különböző lehetséges megoldásokat
vizsgáltunk meg, beleértve általános célú, előre tanított modelleket és
feladatspecifikus, saját fejlesztésű neurális hálózatokat. A végső
megoldás egy konvolúciós neurális hálózaton alapuló szegmentációs modell,
amelyet kifejezetten az adott adathalmaz sajátosságaihoz igazítottunk.


A hagyományos képfeldolgozási módszerek korlátozott hatékonyságot
mutatnak ilyen komplex környezetben, ezért az utóbbi években egyre
nagyobb hangsúlyt kapnak a mélytanulás-alapú megközelítések.
Jelen dolgozat célja egy konvolúciós neurális hálózaton alapuló modell
bemutatása és értékelése az Airbus Ship Detection Challenge adathalmazán,
amely alkalmas hajók pontos lokalizálására műholdképeken.

\section{Tématerület ismertetése és korábbi megoldások}

\subsection{Hajódetektálás műholdképeken}
A hajók automatikus detektálása műholdképeken a számítógépes látás és a
távérzékelés egyik aktívan kutatott területe. A feladat sajátossága,
hogy a képek nagy területet fednek le, miközben a detektálandó objektumok
gyakran kis méretűek és alacsony kontrasztúak. Emellett a háttér erősen
változó lehet, amely tovább nehezíti a megbízható felismerést.

A korai megoldások jellemzően kézzel tervezett jellemzőkön és klasszikus
képfeldolgozási eljárásokon alapultak, azonban ezek teljesítménye
korlátozott volt komplex vizuális környezetben. A mélytanulás
megjelenésével a konvolúciós neurális hálózatok váltak a domináns
megközelítéssé, amelyek képesek a releváns vizuális mintázatok
automatikus kinyerésére.

\subsection{Általános célú szegmentációs modellek}
A projekt kezdeti szakaszában megvizsgáltuk az általános célú,
előre tanított szegmentációs modellek alkalmazhatóságát. Külön figyelmet
fordítottunk a Segment Anything Model (SAM) architektúrára, amely
nagyméretű tanítóadatokon betanítva képes különböző objektumok
szegmentálására minimális felhasználói beavatkozás mellett.

A SAM dokumentációjának és kapcsolódó publikációinak áttekintése alapján
a modell erőssége az általános vizuális reprezentációk kezelése.
Ugyanakkor a műholdképek speciális jellemzői – például a hajók kis
mérete, a változó felbontás és az alacsony kontraszt – miatt a modell
alkalmazása ebben a konkrét feladatban korlátozott hatékonyságot
mutathat. Emellett a modell számítási igénye és komplexitása is
jelentősnek bizonyult.

Ezeket figyelembe véve végül nem egy általános célú szegmentációs
modellt, hanem egy feladatspecifikus, saját tanítású neurális hálózatot
választottunk, amely jobban illeszkedik az Airbus Ship Detection
Challenge adathalmazához.

\subsection{Kaggle közösségi megoldások}
A modell tervezése során a Kaggle platformon elérhető nyilvános
megoldásokat és kódrészleteket is elemeztük. Ezek a közösségi megoldások
értékes betekintést nyújtottak az adathalmaz tipikus problémáiba,
valamint a gyakran alkalmazott architektúrákba és tanítási stratégiákba.

Számos sikeres megoldás U-Net alapú szegmentációs architektúrát
alkalmazott, különböző encoder struktúrákkal és speciális
veszteségfüggvényekkel. Gyakori megközelítés volt a Dice-alapú loss
függvények használata, valamint az adatok kiegyensúlyozásának kezelése,
mivel a hajó nélküli képek jelentős túlsúlyban vannak az adathalmazban.

A közösségi megoldások tanulmányozása inspirációként szolgált, azonban
a végső modell nem kész megoldások átvételén alapult, hanem a
tapasztalatok felhasználásával kialakított, saját implementáció
eredménye.



\section{Rendszerterv}
\section{Rendszerterv}
A megvalósított rendszer egy végponttól végpontig tanítható,
mélytanulás-alapú feldolgozási láncot valósít meg, amely a nyers
műholdképekből közvetlenül hajókat jelölő szegmentációs maszkokat állít
elő. A rendszer célja, hogy minimalizálja a kézi beavatkozás szükségességét,
és automatikusan tanulja meg azokat a vizuális jellemzőket, amelyek a
hajók elkülönítéséhez szükségesek.

A bemeneti adatokat RGB műholdképek alkotják, amelyek különböző
földrajzi területekről és eltérő környezeti feltételek mellett készültek.
A kimenet egy bináris szegmentációs maszk, amely pixel-szinten jelöli a
hajókhoz tartozó területeket. Ez a megközelítés lehetővé teszi nemcsak a
hajók jelenlétének detektálását, hanem azok pontos lokalizációját is.

A rendszer központi eleme egy konvolúciós neurális hálózat, amely
encoder–decoder architektúrát követ. Az encoder feladata a bemeneti
képek hierarchikus jellemzőinek kinyerése, amely során a hálózat egyre
absztraktabb reprezentációkat tanul meg. A decoder ezekből a
reprezentációkból állítja elő az eredeti felbontású szegmentációs maszkot.

Az architektúrában alkalmazott áthidaló kapcsolatok (skip connections)
kulcsszerepet játszanak a finom részletek megőrzésében. Ezek a kapcsolatok
lehetővé teszik, hogy az encoder korai rétegeiben kinyert alacsony szintű
jellemzők közvetlenül eljussanak a decoder megfelelő rétegeibe, ezáltal
javítva a szegmentáció pontosságát, különösen kis méretű objektumok
esetén.


\section{Megvalósítás}
\subsection{Adatok beszerzése és előkészítése}
A feladat során az Airbus Ship Detection Challenge keretében
nyilvánosan elérhető Kaggle adathalmazt használtuk. Az adathalmaz
nagyfelbontású RGB műholdképeket tartalmaz, amelyekhez hajókat jelölő
annotációk tartoznak. Az annotációk futamhossz-kódolással (Run-Length
Encoding, RLE) vannak reprezentálva, amely tömör formában írja le a
bináris szegmentációs maszkokat.

Az előfeldolgozás első lépéseként az RLE formátumú annotációkat
pixel-alapú bináris maszkokká alakítottuk. Ezt követően a bemeneti
képeket normalizáltuk, hogy a pixelértékek egységes tartományba
essenek, ezáltal stabilabb tanítási folyamatot biztosítva. Az adatok
előkészítése során figyelembe vettük azt a sajátosságot is, hogy az
adathalmaz jelentős részében nem található hajó, ami osztály-arány
eltolódáshoz vezethet.


\subsection{Tanítás}
A hálózat tanítása felügyelt tanulási módszerrel történt, ahol a bemeneti
műholdképekhez tartozó bináris maszkok szolgáltak célértékként. A
tanítás során olyan veszteségfüggvényt alkalmaztunk, amely kifejezetten
alkalmas szegmentációs feladatokra, és képes kezelni a hajó és nem hajó
pixelek közötti egyensúlytalanságot.

Az optimalizálást iteratív, gradiens-alapú módszerrel végeztük, amely
során a hálózat paraméterei lépésről lépésre módosultak a veszteség
minimalizálása érdekében. A tanítási folyamat során az adathalmazt
külön tanító és validációs halmazra osztottuk, ami lehetővé tette a
modell általánosítási képességének nyomon követését.

A validációs halmazon mért teljesítmény alapján következtettünk a
túltanulás megjelenésére, és ennek megfelelően állítottuk be a tanítás
hosszát. Ez a megközelítés hozzájárult egy stabilabb és megbízhatóbb
modell kialakításához.


\subsection{Kiértékelés és tesztelés}
A modell teljesítményét a tanítási, validációs és teszt halmazon
értékeltük. A kiértékelés során nemcsak a veszteségfüggvény alakulását
vizsgáltuk, hanem a szegmentáció minőségét jellemző mérőszámokat is,
amelyek pontosabb képet adnak a modell gyakorlati használhatóságáról.

A tanítási és validációs görbék elemzése alapján megállapítható volt,
hogy a modell képes volt megtanulni a releváns vizuális mintázatokat,
miközben elkerülte a jelentős túltanulást. A teszt halmazon elért
eredmények azt mutatták, hogy a hálózat összetett háttérrel rendelkező
képek esetén is megbízhatóan képes a hajók detektálására.

A kiértékelés során külön figyelmet fordítottunk a kis méretű hajók
szegmentációjára, mivel ezek jelentik a feladat egyik legnagyobb
kihívását. Az eredmények alapján a választott architektúra megfelelő
kompromisszumot nyújtott a pontosság és a számítási igény között.



\section{Jövőbeli tervek és összefoglalás}
A bemutatott megoldás igazolja, hogy a konvolúciós neurális hálózatokon
alapuló szegmentáció hatékony eszközt jelent a hajók automatikus
detektálására műholdképeken. A jövőben a modell továbbfejleszthető
komplexebb architektúrák alkalmazásával, valamint kiterjedtebb
adataugmentációs technikák bevezetésével.

További lehetséges irányt jelent az előtanított hálózatok használata,
amelyek gyorsabb konvergenciát és jobb általánosítást eredményezhetnek.
Emellett a módszer más távérzékelési feladatokra, például repülőgépek
vagy épületek detektálására is kiterjeszthető.


\section*{References}
\small

[1] Alexander, J.A.\ \& Mozer, M.C.\ (1995) Template-based algorithms
for connectionist rule extraction. In G.\ Tesauro, D.S.\ Touretzky and
T.K.\ Leen (eds.), {\it Advances in Neural Information Processing
  Systems 7}, pp.\ 609--616. Cambridge, MA: MIT Press.

[2] Bower, J.M.\ \& Beeman, D.\ (1995) {\it The Book of GENESIS:
  Exploring Realistic Neural Models with the GEneral NEural SImulation
  System.}  New York: TELOS/Springer--Verlag.

[3] Hasselmo, M.E., Schnell, E.\ \& Barkai, E.\ (1995) Dynamics of
learning and recall at excitatory recurrent synapses and cholinergic
modulation in rat hippocampal region CA3. {\it Journal of
  Neuroscience} {\bf 15}(7):5249-5262.

\end{document}