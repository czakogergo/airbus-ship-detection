\documentclass{article}

% if you need to pass options to natbib, use, e.g.:
% \PassOptionsToPackage{numbers, compress}{natbib}
% before loading nips_2016
%
% to avoid loading the natbib package, add option nonatbib:
% \usepackage[nonatbib]{nips_2016}

% \usepackage{nips_2016}

% to compile a camera-ready version, add the [final] option, e.g.:
\usepackage[final]{nips_2016}



\usepackage[utf8]{inputenc} % allow utf-8 input
\usepackage[T1]{fontenc}    % use 8-bit T1 fonts
\usepackage{hyperref}       % hyperlinks
\usepackage{url}            % simple URL typesetting
\usepackage{booktabs}       % professional-quality tables
\usepackage{amsfonts}       % blackboard math symbols
\usepackage{nicefrac}       % compact symbols for 1/2, etc.
\usepackage{microtype}      % microtypography

\title{
Ship Detection in Satellite Images Using Deep Learning\\
\large Hajók detektálása műholdképeken mélytanulás segítségével
}



% The \author macro works with any number of authors. There are two
% commands used to separate the names and addresses of multiple
% authors: \And and \AND.
%
% Using \And between authors leaves it to LaTeX to determine where to
% break the lines. Using \AND forces a line break at that point. So,
% if LaTeX puts 3 of 4 authors names on the first line, and the last
% on the second line, try using \AND instead of \And before the third
% author name.

\author{
Sági Benedek \\
Budapest Műszaki és Gazdaságtudományi Egyetem \\
\texttt{benedek.sagi@edu.bme.hu}
\And
Czakó Gergő \\
Budapest Műszaki és Gazdaságtudományi Egyetem \\
\texttt{gergo.czako@edu.bme.hu}
}

\begin{document}
% \nipsfinalcopy is no longer used

\maketitle

\begin{abstract}
\textbf{Abstract}
The automatic analysis of satellite imagery plays a crucial role in maritime traffic monitoring, environmental protection, and security applications. This paper addresses the task defined in the Airbus Ship Detection Challenge, which aims to detect ships in high-resolution satellite images. The proposed approach employs a convolutional neural network-based segmentation model to identify ship locations using binary pixel-level masks. The model is trained in a supervised manner on annotated satellite images, where ship masks are provided in run-length encoded format. During training, separate training and validation sets are used to monitor generalization performance and prevent overfitting. The model is evaluated on training, validation, and test datasets using segmentation-specific metrics and loss functions. Experimental results demonstrate that the deep learning-based segmentation approach is effective for ship detection in complex and noisy visual environments, making it a suitable solution for maritime remote sensing applications.

\vspace{0.7em}
\textbf{Magyar kivonat}
A műholdfelvételek automatikus elemzése kulcsfontosságú szerepet tölt
be a tengeri forgalom megfigyelésében, a környezetvédelemben és a
biztonsági alkalmazásokban. A dolgozat az Airbus Ship Detection
Challenge keretében meghatározott feladatot vizsgálja, amelynek célja
hajók detektálása nagy felbontású műholdképeken. A bemutatott módszer
egy konvolúciós neurális hálózaton alapuló szegmentációs modellt
alkalmaz, amely bináris, pixel-szintű maszkok segítségével határozza
meg a hajók elhelyezkedését a képeken. A modellt felügyelt tanulási módszerrel, annotált műholdképeken
tanítottuk, ahol a hajókhoz tartozó maszkok futamhossz-kódolt
formátumban álltak rendelkezésre. A tanítás során külön tanító és
validációs adathalmazt használtunk az általánosítási képesség nyomon
követésére és a túltanulás elkerülésére. A modell teljesítményét
tanítási, validációs és teszt adathalmazokon értékeltük, szegmentációs
feladatokra jellemző mérőszámok és veszteségfüggvények alkalmazásával. A kísérleti eredmények azt mutatják, hogy a mélytanulás-alapú
szegmentációs megközelítés hatékonyan alkalmazható a hajók
detektálására összetett és zajos vizuális környezetben, így alkalmas
megoldást jelent a tengeri távérzékelési alkalmazások számára.
\end{abstract}



\section{Introduction}
Satellite remote sensing plays a crucial role in the continuous
observation of the Earth’s surface. High-resolution satellite imagery
enables the analysis of maritime traffic, the monitoring of shipping
routes, and the detection of illegal activities, such as unauthorized
fishing. These applications require accurate and reliable automatic
ship detection, which represents a significant challenge due to
highly variable environmental conditions.

The appearance of ships in satellite images is highly heterogeneous:
their size, shape, orientation, and contrast can vary considerably,
while the background often contains noisy water surfaces or coastal
structures. Traditional image processing methods generally lack
robustness in such complex visual environments, which has led to the
increasing adoption of deep learning-based approaches in recent years.

This paper addresses the task defined in the Airbus Ship Detection
Challenge, which focuses on pixel-level ship segmentation in satellite
images. Several possible approaches were considered during the design
phase, including general-purpose pre-trained models and task-specific
neural networks. The final solution is based on a convolutional neural
network-based segmentation model specifically adapted to the
characteristics of the given dataset.

Traditional image processing techniques show limited effectiveness in
such complex scenarios; therefore, deep learning-based methods have
become increasingly prominent. The goal of this work is to present and
analyze a convolutional neural network-based model applied to the
Airbus Ship Detection Challenge dataset, capable of accurately
localizing ships in satellite images.


\section{Related Work and Background}

\subsection{Ship Detection in Satellite Images}
Automatic ship detection in satellite imagery is an actively researched
topic at the intersection of computer vision and remote sensing. A key
challenge of this task is that images often cover large geographic
areas, while the objects of interest are relatively small and exhibit
low contrast. Additionally, the background can vary significantly,
which further complicates reliable detection.

Early approaches typically relied on handcrafted features and
classical image processing techniques; however, their performance was
limited in complex visual environments. With the advent of deep
learning, convolutional neural networks have become the dominant
approach, as they are capable of automatically extracting relevant
visual features from data.

\subsection{General-Purpose Segmentation Models}
During the initial phase of the project, the applicability of
general-purpose, pre-trained segmentation models was examined. Special
attention was given to the Segment Anything Model (SAM), which is
trained on large-scale datasets and is capable of segmenting various
objects with minimal user interaction.

Based on the review of the SAM documentation and related publications,
the model’s main strength lies in handling general visual
representations. However, due to the specific characteristics of
satellite imagery—such as the small size of ships, varying resolution,
and low contrast—the effectiveness of the model for this particular
task may be limited. Furthermore, the computational complexity and
resource requirements of the model were also considered significant.

Taking these factors into account, a task-specific neural network
trained from scratch was selected instead, as it better aligns with
the characteristics of the Airbus Ship Detection Challenge dataset.

\subsection{Kaggle Community Solutions}
During the design of the model, publicly available solutions and code
examples on the Kaggle platform were also analyzed. These community
solutions provided valuable insights into common challenges of the
dataset, as well as frequently used architectures and training
strategies.

Many successful approaches employed U-Net-based segmentation
architectures combined with different encoder structures and
specialized loss functions. Handling class imbalance was a common
strategy, as images without ships are present in significantly larger
numbers within the dataset.

The analysis of community solutions served as inspiration rather than
direct implementation. The final model was developed independently,
using the acquired insights to guide architectural and training
decisions.


\section{System Design}
The implemented system represents an end-to-end trainable deep
learning-based processing pipeline that produces ship segmentation
masks directly from raw satellite images. The primary goal of the
system is to minimize manual intervention while automatically learning
the visual features required to distinguish ships from the background.

The input consists of RGB satellite images captured under varying
geographical and environmental conditions. The output is a binary
segmentation mask indicating ship pixels at the pixel level. This
approach enables not only the detection of ship presence but also their
precise localization.

The core component of the system is a convolutional neural network
following an encoder–decoder architecture. The encoder extracts
hierarchical feature representations from the input images, learning
increasingly abstract features. The decoder reconstructs the
segmentation mask from these representations at the original image
resolution.

Skip connections play a crucial role in preserving fine-grained
details by allowing low-level features from the encoder to be directly
propagated to corresponding decoder layers, thereby improving
segmentation accuracy, particularly for small objects.


\section{Implementation}
\subsection{Data Acquisition and Preprocessing}
The experiments were conducted using the publicly available Airbus
Ship Detection Challenge dataset hosted on Kaggle. The dataset consists
of high-resolution RGB satellite images along with corresponding ship
annotations. The annotations are provided in run-length encoded (RLE)
format, which compactly represents binary segmentation masks.

As a first preprocessing step, the RLE annotations were converted into
pixel-level binary masks. The input images were then normalized to
ensure consistent pixel value ranges, contributing to a more stable
training process. Additionally, the significant class imbalance caused
by the large number of images without ships was taken into account
during data preparation.


\subsection{Training}
The network was trained using supervised learning, where binary
segmentation masks served as target labels for the corresponding input
images. A loss function suitable for segmentation tasks was employed,
with the ability to handle the imbalance between ship and background
pixels.

Optimization was performed using an iterative gradient-based method,
where network parameters were updated to minimize the loss function.
The dataset was split into training and validation subsets, enabling
continuous monitoring of the model’s generalization behavior during
training.

Based on validation performance, training duration was adjusted to
mitigate overfitting. This approach contributed to the development of
a more stable and reliable model.


\subsection{Evaluation}
Model performance was assessed using training and validation subsets.
The primary goal of evaluation was to monitor the training process and
analyze model behavior, with particular focus on the evolution of the
loss function during training.

Analysis of training and validation loss values indicated a gradual
reduction in training error, while validation loss remained relatively
stable. This suggests that the model was able to learn meaningful
visual features without relying solely on memorization of training
data.

Qualitative evaluation was performed by visually inspecting the
segmentation masks generated for validation images. The results showed
that the model was able to identify ship outlines in many cases,
although inaccuracies occurred, particularly for small, low-contrast
objects or images containing land areas. Overall, the evaluation
demonstrates that the chosen architecture is suitable for addressing
the core aspects of the task, while still leaving room for further
improvements.


\section{Future Work and Conclusion}
The presented results demonstrate that convolutional neural
network-based segmentation is an effective approach for automatic ship
detection in satellite imagery. Future work may include the application
of more complex architectures and the incorporation of more extensive
data augmentation techniques.

Another promising direction is the use of pre-trained networks, which
could lead to faster convergence and improved generalization.
Additionally, the proposed approach could be extended to other remote
sensing tasks, such as the detection of aircraft or buildings.


\section*{References}
\small

[1] Alexander, J.A.\ \& Mozer, M.C.\ (1995) Template-based algorithms
for connectionist rule extraction. In G.\ Tesauro, D.S.\ Touretzky and
T.K.\ Leen (eds.), {\it Advances in Neural Information Processing
  Systems 7}, pp.\ 609--616. Cambridge, MA: MIT Press.

[2] Bower, J.M.\ \& Beeman, D.\ (1995) {\it The Book of GENESIS:
  Exploring Realistic Neural Models with the GEneral NEural SImulation
  System.}  New York: TELOS/Springer--Verlag.

[3] Hasselmo, M.E., Schnell, E.\ \& Barkai, E.\ (1995) Dynamics of
learning and recall at excitatory recurrent synapses and cholinergic
modulation in rat hippocampal region CA3. {\it Journal of
  Neuroscience} {\bf 15}(7):5249-5262.

\end{document}